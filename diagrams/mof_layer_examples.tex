%\documentclass[review]{elsarticle}
\documentclass[]{elsarticle}
\usepackage{underscore}
\usepackage{tikz}
\usepackage{tikz-uml}
\usetikzlibrary{positioning}
\usetikzlibrary{calc}

\usepackage{local-tikz-style}

%%\tikzstyle{tikzuml inherit style}=[color=\tikzumlDefaultDrawColor, triangle 90-open triangle 45]%
%%\renewcommand{\umlinherit}[3][]{\umlrelation[style={tikzuml inherit style}, #1]{#2}{#3}}%

\tikzset{
  sc1/.style = {
    minimum width = 3cm,
    minimum height = 0.3cm,
    %%execute at begin node=\setlength{\baselineskip}{0em},
    inner sep=1pt,
    anchor=south
    },
  cl2/.style = {cl1,
    anchor=west,
    },
  note1/.style = {inner sep=4pt, align=left},
  note2/.style = {inner sep=5pt, align=left},
}

\begin{document}


%%HAS REGISTRY BEEN SKIPPED HERE?
\begin{figure}[p]
\begin{tikzpicture}

%% don't know why anchor is not being set here ...
\begin{scope}[shift={(0.,8cm)}]
\umlclass[cl2,anchor=west,type={division},name=tech]{tech}{}{}
\umlnote[note1, x=7cm, width=5cm]{tech}{Other divisions are {organisation} and {community}}
\umlnote[note2, x=-3cm, width=4cm]{tech}{This layer describes the three main divisions of work};
\end{scope}

\begin{scope}[shift={(0.,6cm)}]
\umlsimpleclass[cl2,anchor=west,type={tech}]{register}
\umlnote[note1, x=7cm, width=5cm]{register}{Other components are {software} and {services}}
\end{scope}

\begin{scope}[shift={(0.,4cm)}]
\umlsimpleclass[cl2,anchor=west,type={register}]{variable}
\umlnote[note1, x=7cm, width=5cm]{variable}{Other components are, for example, {units} and {grids}}
\end{scope}

\begin{scope}[shift={(0.,0cm)}]
\umlclass[cl2,x=-1cm,y=0, anchor=west,type={attribute}]{variable::units}{id="....."\\title:"Units of Measure"\\
                      description="....."\\standard_name : "air_temperature"\\.....}{}

\end{scope}
\begin{scope}[shift={(0.,-4cm)}]
\umlclass[cl2,x=-1cm,y=0, anchor=west, type={variable}]{tas}{id="....."\\title:"Near Surface Air Temperature"\\
                      units="K"\\
                      description="....."\\standard_name : "air_temperature"\\.....}{}
\end{scope}



\end{tikzpicture}
\caption{MOF layers}
\label{fig:mofl}
\end{figure}

The 6 layers may appear excessive for a small project, but each layer is introducing significant generalizations.

Instead, adopt the approach that a package has a parent class in the MOF layer above, and contains a collection of classes. This allows two organisational layers to be placed within a MOF layer. We could, of course, compress it to a single layer and use lots of sub-packages to organise the work, but the MOF layering allows some conceptual clarity. 

The base layer (M0) contains the the records of the meta data registry.
The layer above (M1) describes the generic record by defining attributes used within each table.

M2 describes the generic table, in terms of the attributes of each table and the syntax of links between tables.

M3 describes the registry as a whole, and other information technology components, particularly the python library
which supports use of the information in the registers. There are considerable dependencies between these
two components.

M4 describes the three divisions of activity. The organisation, including staff and resources, the community, including the 
networks engaged in setting standards, and the information technology.

M5, finally, is the top MOF layer which describes the modeling approach.

{\it
There are issues with the MOF abstraction layer approach when applied to organisations. The layers should be
functionally independent, but the teams and individuals identified in the organisation are defined at the lower 
level .... need to be clear about path of instantiation}

There is no uniquely correct approach to defining MOF layers. The rule that the top layer should define the approach used 
in other layers gives considerable freedom.

The process of instantiation is intended to be a procedural rather than executive one, bringing about 
what has been defined in the classes of the parent layer. Instantiation does, however, involve some inputs.

Instantiation of M4 requires the specification of:
\begin{itemize}
\item A profile of ISO 11179 for the information technoloy division;
\item A profile of ISO 19440 for the organisation and community engagement;
\item Specification of interfaces between these elements;
\end{itemize}

Instantiation of M4 also involves setting out the principle areas of action 
in each division, defined as packages, and setting out the objectives and primary resources of each of these packages.

Instantiation of M3 then takes a slightly different form in each division or package. Setting out the implementation
decisions, such as choice of languages and external resources (libraries, technical standards, etc).

\begin{itemize} 
\item For information technology, the specification of class factories which
\end{itemize}

Instantiation of M2: for the python API, this involves defining the core meta-class (implemented as class a factory). For the
registry, it involves defining the registers. For the community and organisation sections, this involves definition of the
governance processes for the communication and implementation. Here, the ... concept of multi-level, devolved governance is
used, with governance bodies defined at the M2 level reporting back to higher level bodies (the fact that higher level
governance bodies are defined at higher MOF levels is a feature of this implementation ... the level of
governance and level in the MOF hierarchy are not intrinsically linked ab initio).

Instantiation of M1 then involves defining python classes or database records.

In terms of ISO 19439, the levels can be identified with:
\begin{itemize} 
\item M3: domain identification;
\item M2: concept definition;
\item M1: requirements definition;
\item M0: design specification;
\end{itemize} 


Key parts of 19440 are decision centres, functional entities, organizational units, and person profiles.  Decision centres provide
a basic documentation of the agency that can trigger the change of status of an item in the registry.

For each register there must be one or more organisational units responsible for items in the register, and there must be decision
centres assigned to enact the necessary status changes of items.
In plain terms, we identify groups or bodies who can propose to a register and make decisions. This will include harmonisation
roles as well as groups or networks providing community requirements.

The 19440 and 11179 standards take complimentary approaches. 19440 describes information content requirements, 11179 describes a modelling framework.

This, we might consider the 19440 Person Profile and a 11179 Object, with Property classes defined for Design Authority, Role, Organizational Skills, Operational Skills, Provider of Skills, etc. This mapping will allow a model of the organisational structure behind the decision processes
of the registry to be represented in the registry.

MDR_Registered_Item needs to conform to EIC_Product (19440). This could be done through an association of classes, but
here an isomorphism is proposed, through a stereotype of MDR_Registered_Item which conforms to EIC_Product.

\begin{table}[h]
\begin{tabular}{|l|l|p{7cm}|}
\hline
Registration Authority & Design Authority & References and EIC_Organizational Unit, or an MDR_Organization: the latter a light-weight basic class with 5 text attributes, the former has a signifcant amount of detail. This detail will be useful in representing the relationship between
MIPs, groups within MIPs and projects which link several MIPs.  \\
& Description & \\
& Nature of Product & has value "INFORMATION" \\
& Properties &  set of concepts  \\
& Constraints & Constraint on an object \\
& Integrity Rules &  Constraints on attributes\\
\hline
\end{tabular}
\caption{mapping DR_Registered_Item to EIC_Product}
\end{table}


The Design Authority for a product will be the Registration Authority, carrying the specific responsibility for ensuring all processes required
are complete before registration. There is one unique EIC_Design_Authority for each EIC_Product. Within the registry, there will be multiple Registration_authority instances associated with different stakeholders. 

The MDR_Property is a basic concept. Attributes in DREQ are data elements, in the sense that they have a defined meaning and a value domain.
These can become an identified item with a value specified in a slot.

\subsection{The self-referential bit of DREQ}

The attributes which are used to describe attributes are characterised the fact that they play both the role of MDR_Object_Class and MDR_Property.

As an MDR_Object, attribute.title has properties attribute.title, attribute.label, etc.

For a register record, such as "label:tas, title:Near-surface Air Temperature, ", there is no MDR_Data_Element, just a "Data_element_concept".


Registered Item .. can be "Concept" or (is a "Data Element" a concept?)



\end{document}
