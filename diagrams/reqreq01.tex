 
% Author: Martin Juckes (based on work from Qrrbrbirlbel)
% pdflatex -shell-escape mgtchart.tex for fancy NERC Research font
\documentclass[tikz,border=10pt]{standalone}
%%%<
\usepackage{verbatim}
%%%>

\usepackage{helvet}
\renewcommand{\familydefault}{\sfdefault}
\renewcommand{\rmdefault}{phv} % Arial
\renewcommand{\sfdefault}{phv} % Arial

%%\usepackage{auto-pst-pdf}% for running it with pdflatex -shell-escape
\usepackage{pst-light3d}
%%\DeclareFixedFont{\Sf}{T1}{fxb}{b}{n}{16pt}
\DeclareFixedFont{\Sf}{T1}{phv}{b}{n}{16pt}

\begin{comment}
:Title: Flowchart
:Tags: Diagrams;Flowcharts;Node positioning;Nodes and shapes
:Author: Qrrbrbirlbel
:Slug: flowchart

This example flowchart uses the positioning-plus library and the
node-families library by the same author. The fit library is implicitly loaded
by positioning-plus, the backgrounds library is used to draw stuff behind
other stuff, the calc library for some coordinate calculations and the
shapes.geometric library for the ellipse shape.

This example was written by Qrrbrbirlbel answering a question on TeX.SE.
\end{comment}
\usetikzlibrary{shapes.geometric,backgrounds,
  positioning-plus,node-families,calc}
% for chamfered rectangle
\usetikzlibrary{shapes.misc,shapes.symbols}
\tikzset{
  clear box/.style = {
    shape = rectangle,
    align = center,
    draw  = #1,
    rounded corners},
  basic box/.style = {
    shape = rectangle,
    align = center,
    draw  = #1,
    fill  = #1!25,
    rounded corners},
  management func/.style = {signal, draw, thick, 
               signal to=west and east,
               text width=22mm, minimum height=9mm, align=center},
  wp box/.style = {
    shape = rectangle,
    align = center,
    draw  = blue,
    minimum width=1.1cm,
    fill  = blue!25!yellow,
    rounded corners},
  wp ellipse/.style = {
    shape = ellipse,
    inner sep = 0,
    minimum size = 1pt,
    align = center,
    draw  = blue,
    x radius=0.9cm,
    y radius=0.4cm,
    fill  = blue!25!yellow
    },
  dc box/.style = {
    shape = chamfered rectangle,
    align = center,
    draw  = blue,
    minimum width=2.5cm,
    minimum height=4.2cm,
    fill  = green!25!yellow},
  rc box/.style = {
    shape = ellipse,
    align = center,
    draw  = blue,
    minimum width=2.5cm,
    fill  = red!25,
    rounded corners},
  header node/.style = {
    Minimum Width = header nodes,
    %%font          = \strut\Large\ttfamily,
    font          = \strut\large,
    text depth    = +0pt,
    fill          = white,
    anchor = west,
    draw},
  wpheader node/.style = {
    font          = \strut,
    text depth    = +0pt,
    fill          = white,
    draw},
  dclabel/.style = {%
    inner ysep = +1.5em,
    append after command = {
      \pgfextra{\let\TikZlastnode\tikzlastnode}
      node [wpheader node] (lab-\TikZlastnode) at (\TikZlastnode.south) {#1}
    }
    },
  wpheader/.style = {%
    inner ysep = +1.5em,
    append after command = {
      \pgfextra{\let\TikZlastnode\tikzlastnode}
      node [wpheader node] (header-\TikZlastnode) at (\TikZlastnode.north east) {#1}
    }
  },
  header/.style = {%
    inner ysep = +1.5em,
    append after command = {
      \pgfextra{\let\TikZlastnode\tikzlastnode}
      node [header node] (header-\TikZlastnode) at (\TikZlastnode.north west) {#1}
      node [span = (\TikZlastnode)(header-\TikZlastnode)]
        at (fit bounding box) (h-\TikZlastnode) {}
    }
  },
  hv/.style = {to path = {-|(\tikztotarget)\tikztonodes}},
  vh/.style = {to path = {|-(\tikztotarget)\tikztonodes}},
  fat red line/.style = {ultra thick, red, rounded corners},
  fat blue line/.style = {ultra thick, blue, rounded corners}
}
\newlength{\vpathoff}
\setlength{\vpathoff}{0.3cm}
\newlength{\hpo}
\setlength{\hpo}{0.3cm}
\newlength{\po}
\setlength{\po}{0.3cm}
\newlength{\poc}
\setlength{\poc}{0.05cm}
\newlength{\dcsk}
\setlength{\dcsk}{0.4cm}
\newlength{\wpsk}
\setlength{\wpsk}{0.8cm}
\begin{document}

  \pgfdeclarelayer{base}
  \pgfdeclarelayer{back}
  \pgfdeclarelayer{mid}
  \pgfdeclarelayer{fore}
  \pgfsetlayers{base,back,mid,fore}
\begin{tikzpicture}[node distance = 1.2cm, thick, nodes = {align = center},
    >=latex]

\begin{pgfonlayer}{fore}
  \node[wp box, wpheader=WP1] (wp1) {Documentation};
  \node[wp box, wpheader=WP2, right = \wpsk of wp1] (wp2) {Software};
  \node[wp box, wpheader=WP3, below = .7cm of wp1.south west, anchor = north west] (wp3) {Standards};

  \node[dc box, wpheader=CMIP, below = 4.6cm of wp3.west, anchor=west] (reg1) {};
  \node[dc box, right = \dcsk of reg1.east, anchor = west]  (reg2)  {};
  \node[dc box, right = \dcsk of reg2.east, anchor = west]  (reg3)  {};
  \node[wp box, below = .5cm of reg1.north, anchor=north] (cmip) {config.};
  \node[wp box, below = .5cm of cmip, anchor=north] (cmip) {content};
  \node[wp box, wpheader=Reference, below = .1cm of reg2.north, anchor=north] (ref) {};
  \node[wp box, wpheader=H2020, below = .1cm of reg3.north, anchor=north] (h2020) {};

  \begin{scope}
    \node[fit = {(reg1.west)(header-h2020.north east)(reg3)}, clear box = blue,
      header = Registers] (registers) {};
  \end{scope}
   
\end{pgfonlayer}{fore}

\begin{pgfonlayer}{mid}
  \begin{scope}
    \node[fit = {(wp1)(header-wp2)(wp3)}, clear box = blue,
      header = Framework] (framework) {};
  \end{scope}
  \node[management func, above = 1cm of framework.north east, anchor = south east] (oper) {Operations};
\end{pgfonlayer}{mid}

\begin{pgfonlayer}{back}
  \begin{scope}
    \node[fit = {(framework)(registers)(oper)}, basic box = blue, inner sep=10pt,
      header = Hosted Registers] (hosted) {};
  \end{scope}
  \node[management func, above = 0.7cm of hosted.north, anchor = south] (mgt) {Registry\\Management};
\end{pgfonlayer}{back}

\begin{pgfonlayer}{base}
    \node[fit = {($(hosted.north west)+(0,20pt)$)(mgt)(oper)(hosted.south east)},
             basic box = orange, inner sep=10pt,
      header = Registry] (registry) {};
\end{pgfonlayer}{base}

\begin{pgfonlayer}{mid}

  \node[rc box, left = 1.5cm of reg1] (wip) {WIP};
  \node[rc box, above = of wip] (dtf) {DTF};

  \node[management func, above = 0.4cm of registry] (ceda) {CEDA};


\end{pgfonlayer}{mid}

\begin{pgfonlayer}{fore}


  \path[fat blue line, dashed, vh] (wp2.east) -- (ceda.east);

  \path let \p1=(wip.west), \p2=(registry.south) in coordinate (caa) at ($(\x1,\y2)+(-.2cm,-0.6cm)$);
  \path let \p1=(wip.east), \p2=(registry.south) in coordinate (caaa) at ($(\x1,\y2)+(0,-0.6cm)$);
  \path[draw, fat blue line, rounded corners] (reg1.south) |- (caaa) -| (wip.south);
  
  \path[draw, fat blue line, rounded corners] (reg3.south) |- ($(caa)+(-2\po,-3\po)$) |- (dtf.west);
%%
%% Research Centres to NERC
  
\end{pgfonlayer}{fore}
\end{tikzpicture}
\end{document}
