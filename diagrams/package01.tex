\documentclass[review]{elsarticle}
\usepackage{tikz}
\usepackage{tikz-uml}
\usetikzlibrary{positioning}
\usetikzlibrary{calc}

\usepackage{local-tikz-style}

\usepackage{lineno,hyperref}
\modulolinenumbers[5]

\journal{Journal of \LaTeX\ Templates}

%%%%%%%%%%%%%%%%%%%%%%%
%% Elsevier bibliography styles
%%%%%%%%%%%%%%%%%%%%%%%
%% To change the style, put a % in front of the second line of the current style and
%% remove the % from the second line of the style you would like to use.
%%%%%%%%%%%%%%%%%%%%%%%

%% Numbered
%\bibliographystyle{model1-num-names}

%% Numbered without titles
%\bibliographystyle{model1a-num-names}

%% Harvard
%\bibliographystyle{model2-names.bst}\biboptions{authoryear}

%% Vancouver numbered
%\usepackage{numcompress}\bibliographystyle{model3-num-names}

%% Vancouver name/year
%\usepackage{numcompress}\bibliographystyle{model4-names}\biboptions{authoryear}

%% APA style
%\bibliographystyle{model5-names}\biboptions{authoryear}

%% AMA style
%\usepackage{numcompress}\bibliographystyle{model6-num-names}

%% `Elsevier LaTeX' style
\bibliographystyle{elsarticle-num}
%%%%%%%%%%%%%%%%%%%%%%%

\begin{document}

\begin{frontmatter}

\title{Using ISO standards to design a metadata registry for climate data}
%%\title{Elsevier \LaTeX\ template\tnoteref{mytitlenote}}
%%\tnotetext[mytitlenote]{Fully documented templates are available in the elsarticle package on \href{http://www.ctan.org/tex-archive/macros/latex/contrib/elsarticle}{CTAN}.}

%% Group authors per affiliation:
\author{Martin Juckes}
%%\author{Elsevier\fnref{myfootnote}}
\address{Rutherford Appleton Laboratory, Didcot, UK}
%%\fntext[myfootnote]{Since 1880.}

%% or include affiliations in footnotes:
%%\author[mymainaddress,mysecondaryaddress]{Elsevier Inc}
%%\ead[url]{www.elsevier.com}

%%\author[mysecondaryaddress]{Global Customer Service\corref{mycorrespondingauthor}}
%%\cortext[mycorrespondingauthor]{Corresponding author}
%%\ead{support@elsevier.com}

%%\address[mymainaddress]{1600 John F Kennedy Boulevard, Philadelphia}
%%\address[mysecondaryaddress]{360 Park Avenue South, New York}

\begin{abstract}
The climate modelling community collaborate globally to generate a coordinated portfolio of climate simulations which serve to advance scientific understanding and to support the Assessment process of IPCC. 
The interoperability of data products among the participating institutions is guaranteed by a detailed specification of the parameters to be archived and the associated metadata requirements.
This paper looks at the potential for increasing interoperability towards users outside this core community by expressing metadata requirements through the language of ISO standards on metadata registries. 
\end{abstract}

\begin{keyword}
Data registry\sep
Climate data management
\end{keyword}

\end{frontmatter}

\linenumbers




%% standards references not set well as stands ... can modify the titles or look for other styles in article-header.


%% beging document is embedded in article-header
%%\begin{document}

\section{Introduction}\label{sec:intro}

The CMIP6 Data Request provides detailed technical specifications of thousands of climate parameters which are being archived by climate modelling centres around the world as part of a global effort to update the set of reference climate simulations which guide global policy on climate change mitigation and adaptation. 

The DREQ is built on domain standards which have evolved with the CMIP project. In this paper we explore the feasibility and potential benefits associated with expressing
these specifications through ISO standards. 

The work cuts across a range of standards which are introduced in Section \ref{sec:iso} below, covering aspects of geospatial referencing, physical quantities, and the organisation and processes inherent in running a registry of metadata specifications. 

The expected benefits will be both in terms of inter-operability with other standards which deal with environmental information and also in terms of learning from and exploiting practises which are embedded in the ISO standards. 

\subsection{Exploiting the International Standards Organisation [ISO]}\label{sec:iso}

[provisional literature notes]

\cite{sinaci2013} describes the ise of 11179 to facilitate exchange across clinical work and care domains.


The main pillars of the work will be: 

\paragraph{ISO 11179} \citep[Metadata Registry][]{Pon2009,iso80000-1} provides a framework for the organisational structure of a metadata registry and for
the technical specifications of the registers within that registry. Critically, this clarifies the decision processes and responsibilities surrounding 
the registration of new items. 

Here are two sample references: \cite{iso80000-1}.

\cite{iso19135-1} provides information on geospatial referencing.

\section{Overview of the metadata structures}

\begin{figure}[p]
\begin{tikzpicture}


\begin{umlpackage}[fill=green!20]{Metadata}
\begin{umlpackage}[]{Parameters}
\umlsimpleclass[sc1,y=0]{units}
\umlsimpleclass[sc1,y=-0.8cm]{quantities}
\umlsimpleclass[sc1,y=-1.6cm]{constraints}
\umlsimpleclass[sc1,y=-2.4cm]{variables}
\end{umlpackage}

\begin{umlpackage}[]{Data Axes}
\umlsimpleclass[sc1,x=5cm,y=0]{axes}
\umlsimpleclass[sc1,x=5cm,y=-1cm]{coordinates}
\umlsimpleclass[sc1,x=5cm,y=-2cm]{configurations}
\end{umlpackage}

\begin{umlpackage}[]{Data Types}
\umlsimpleclass[sc1,x=5cm,y=-4.6cm]{\detokenize{data-type}}
\umlsimpleclass[sc1,x=5cm,y=-5.4cm]{spatial-grid}
\umlsimpleclass[sc1,x=5cm,y=-6.2cm]{temporal-grid}
\umlsimpleclass[sc1,x=5cm,y=-7cm]{\detokenize{cell-methods}}
\end{umlpackage}

\begin{umlpackage}[]{Request}
\umlsimpleclass[sc1,x=0cm,y=-4.6cm]{variable-group}
\umlsimpleclass[sc1,x=0cm,y=-5.4cm]{experiment-group}
\umlsimpleclass[sc1,x=0cm,y=-6.2cm]{objective}
\begin{umlpackage}[fill=black!10]{Choices}
\umlsimpleclass[sc1,x=0.4cm,y=-8.0cm]{options}
\end{umlpackage}
\end{umlpackage}

\begin{umlpackage}[]{Imports}
\umlsimpleclass[sc1,x=0cm,y=-10.2cm]{MIP}
\umlsimpleclass[sc1,x=0cm,y=-11cm]{standard-name}
\umlsimpleclass[sc1,x=5cm,y=-10.2cm]{experiment}
\end{umlpackage}


\end{umlpackage}

%%% Request .. linking
%%% Experiment ... [obsolete ... only need a class with links to experiments]
%%% Data types (structure).
%%% Data axes (structure).
\end{tikzpicture}
\caption{The metadata package is split into 5 sub-packages characterised by different harmonization and conformance requirements and mechanisms.}
\label{fig:packages}
\end{figure}

\section{Conclusions}\label{sec:conc}

The mapping of the DREQ onto the ISO standards reveals areas where improvements can be made in terms of clarity of decisions making processes and a structured approach to definingthe attributes which characterise registered items.

There does not appear to be any inherent obstacle to full compliance, though it has not been the purpose of this paper to compile a full technical specification.
\section{References}

\bibliography{ams}
\end{document}
